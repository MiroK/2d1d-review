\documentclass[10pt, a4paper]{article}
\usepackage{amsmath, amsfonts, amssymb}
\usepackage{enumerate}
\usepackage{caption, subcaption, floatrow}
\usepackage{graphicx}
\usepackage{multirow}
\usepackage{xfrac}
\usepackage{amsmath, amsfonts, amssymb}
\usepackage{tikz}
\usepackage{pgfplots}
\usepackage{pgfplotstable}
\usetikzlibrary{plotmarks, calc}
\usepackage{standalone}
\usepackage{url}

%%%%%%% COLORS
\usepackage{xcolor, colortbl}
\newcolumntype{S}{>{\columncolor{lime!50}}c}         % serial small
\newcolumntype{L}{>{\columncolor{cyan!50}}c}        % serial large
\newcolumntype{P}{>{\columncolor{red!50}}c}        % parallel& 
%%%%%%%

\newcommand{\reals}{\mathbb{R}}
\newcommand{\set}[1]{\{#1\}}
\newcommand{\abs}[1]{\lvert#1\rvert}
\newcommand{\semi}[1]{\lvert#1\rvert}
\newcommand{\norm}[1]{\lVert#1\rVert}
\newcommand{\inner}[2]{\ensuremath{\left(#1, #2\right)}}
\renewcommand{\brack}[1]{\langle#1\rangle}
\newcommand{\average}[1]{\ensuremath{\langle#1\rangle} }
\newcommand{\jump}[1]{\ensuremath{[\![#1]\!]} }
\newcommand{\mat}[1]{\ensuremath{\mathsf{#1}}}
\newcommand{\dual}[1]{\ensuremath{{#1}^{\prime}}}
\renewcommand{\vec}[1]{\mat{#1}}
\newcommand{\supp}{\operatorname{supp}} 

\oddsidemargin=3pt
\textwidth=500pt
\topmargin=3pt

\title{Notes on $Q$-cap efficiency}

\begin{document}
\maketitle

\section*{Introduction}
To estimate efficiency of the $Q$-cap preconditioner we consider the problem of
spectral equivalence of the $H^{1/2}$ operator with the exact Schur complement of 
system $\left[\left[A, \dual{T}\right], \left[T, 0\right]\right]$. Here $A$ is a 
CG$_1$ discretization of a two-dimensional Laplacian and $T$ is the restriction from
CG$_1$ space setup on two-dimensional to CG$_1$ space over a
one-dimensional domain. This example is useful since it includes construction of 
a preconditioner for the $A$ matrix, calls to preconditioned conjugate gradient 
method in order to invert $A$ and obviously construction of $H^{1/2}$ matrix
representation. These tasks are where most of the time in solving the coupled
multiphysics (plate-beam) problem would be spent. Further the example reveals some
features of the mass lumping - a potential remedy in making the construction of
$H^{1/2}$ more efficient

In the experiment we take \texttt{UnitSquareMesh(n, n)} as the two-dimensional 
domain. The one-dimensional domain is a horizontal line through the center of the 
unit square. As such, the size of $A$ is proportional to $n^2$ while the size of
the Schur complement grows linearly in $n$.

\section*{Eigenvalues}
% Compilation
In order to construct $H^{1/2}$ we solve a generalized eigevalue problem(GEVP) $Bx=\lambda
M x$ with $B$, $M$ the stiffness and mass matrices of CG$_1$ spaces over the
one-dimensional domain. The system is solved via a call to LAPACK.DSYGVD 
routine. Its performance depends heavily on how LAPACK is compiled and unoptimized
versions can lead to a factor ten loss in performance. This can be seen in table
\ref{tab:eig_scale} which compares execution times of Python and Julia programs that
solve the eigenvalue problem. Both languages call LAPACK but only the latter uses
an optimized version of the library. Note that it takes about 20 seconds to
construct the norm operator for
{\setlength{\fboxsep}{0pt}\colorbox{cyan!50}{problems}} of order milion
uknowns ($\approx$16 million) that could still be potentially run in serial.
%
\begin{table}[ht]
  \caption{Exacution times and complexity for GEVP with unoptimized(Python) and 
    optimized(Julia) version of LAPACK. Colors represent the size of the 
    two-dimensional Laplace problem: {\setlength{\fboxsep}{0pt}\colorbox{lime!50}{small scale}}, 
  {\setlength{\fboxsep}{0pt}\colorbox{cyan!50}{large scale (still serial)}} and 
  {\setlength{\fboxsep}{0pt}\colorbox{red!50}{HPC}}.
}
\label{tab:eig_scale}
\footnotesize{
\begin{tabular}{lr|SSSLLLPP}
\multicolumn{2}{l|}{$n=2^i$}    &    7 &    8 &    9 &   10 &   11 & 12     & 13      & 14\\
\hline
Python & $\left[s\right]$       & 0.01 & 0.05 & 0.36 & 3.21 &22.89 & 151.82 & 1032.08 & 7385.45\\
cmplx  &                        & 2.19 & 2.65 & 2.96 & 3.17 & 2.84 & 2.73   & 2.77    & 2.84\\
\hline
Julia  & $\left[s\right]$       & 0.00 & 0.01 & 0.05 & 0.33 & 2.38 & 17.42  & 126.43  & 955.12\\
cmplx  &                        & 1.46 & 1.78 & 2.05 & 2.69 & 2.85 & 2.87   & 2.86    & 2.92\\
\end{tabular}
}
\end{table}
%

% Scaling
Table \ref{tab:eig_scale} shows that LAPACK.DSYGVD is approximately cubic in the
size of the eigenvalue problem. The only feature of the system that the
algorithm exploits is its symmetry. The tridiagonal propery is discarded. This
seems to be a significant loss in the light of the fact that there exist 
$\mathcal{O}(n^2)$ and $\mathcal{O}(n\log{n})$ algorithms for eigenvalue
problems(EVP) with symmetric tridiagonal matrices. The algorithm with quadratic
scaling is provided by LAPACK's DSTEV routine, cf. table \ref{tab:eig_scale_std}.
%

\begin{table}[ht]
\caption{Exacution times and complexity for EVP with optimized LAPACK.DSTEV}
\label{tab:eig_scale_std}
\footnotesize{
\begin{tabular}{l|SSSLLLPP}
$n=2^i$                 &    7 &    8 &    9 &   10 &   11 & 12     & 13      & 14\\
\hline
Julia $\left[s\right]$ & 0.00 & 0.01 & 0.02 & 0.08 & 0.32 & 1.45 & 7.73 & 50.59\\
cmplx                  & 1.79 & 3.09 & 0.81 & 2.09 & 1.99 & 2.19 & 2.42 & 2.71\\
\end{tabular}
}
\end{table}
%

% Better: symtridiagonal but ...
If we were able to solve GEVP in $\mathcal{O}(n^2)$ steps the scaling of the
algorithm would match that of an optimal solver for the system $Ax=b$. However,
routines implementing more efficient algorithms for GEVP with both matrices
being symmetric and tridiagonal are scarce. In this situation we shall try to
transform our GEVP to EVP without damaging properties of the system. 

We first note that $M^{-1} B$ is certainly not sparse. Moreover it is not a symmetric 
matrix. One way to keep the tridiagonal property of the transformed GEVP is to lump 
the mass matrix $M$. Let $\hat{M}$ be the lumped mass matrix. Then $\hat{M}^{-1} B$ 
is indeed tridiagonal. Unfortunately the matrix is not symmetric and it remains to 
be seen if there exist efficient algorithms to solve the lumped EVP. That said,
in our application the quest for such routines would be redundant if the lumping
resulted in a norm operator which is not spectrally equivalent with the Schur
complement. Let us there for investigate the effect of the transformation.

\section*{Lumping effect}
Let $S=TA^{-1}\dual{T}$ denote the symmetric positive definite matrix that is the
negative Shur complement of our linear system. Matrix $H^{1/2}$ shall be the
matrix representation of the equally named norm operator constructed from the
generalized eigenvalue problem. Finally $\tilde{H}^{1/2}$ will be the matrix 
constructed from eigenpairs of the eigenvalue problem with the lumped mass
matrix. We remark that we wish to demonstrate that there exist constants $c, C>0$
independent of $n$ such that for all $x\in\reals^n$
%
\begin{equation}\label{eq:equiv}
  c\dual{x} H x \leq \dual{x} S x \leq  C\dual{x} H x.
\end{equation}
Here $H$ is the placeholder for $H^{1/2}$ and $\tilde{H}^{1/2}$ and the
constants for both operators can be different.

Table \ref{tab:lump} shows that both $H^{1/2}$ and $\tilde{H}^{1/2}$ are
spectrally equivalent with $S$. The effect of lumping seems to be that $c$ is
shifted from value 0.2 to approx $1/3$. The upper bound $C$ stays close to
$1/2$. The table also shows that solving the EVP due to lumping is cheaper. 
%
\begin{table}[ht]
  \caption{Spectral equivalence constants \eqref{eq:equiv} for matrices
  $H^{1/2}$ and $\tilde{H}^{1/2}$ stemming respectively from the generalized
eigenvalue problem $Bx=\lambda M x$ and the eigenvalue problem
$\tilde{M}^{-1}Bx=\lambda x$.}
\label{tab:lump}
\footnotesize{
\begin{tabular}{lr|ccccccc}
\multicolumn{2}{l|}{$n=2^i$}              &     5 &     6 &     7 &     8 &     9 & 10    & 11\\
\hline
\multicolumn{2}{l|}{$c, H^{1/2}$}         & 0.205 & 0.204 & 0.204 & 0.204 & 0.204 & 0.204 & 0.204\\
\multicolumn{2}{l|}{$C, H^{1/2}$}         & 0.494 & 0.498 & 0.499 & 0.500 & 0.500 & 0.500 & 0.500\\
GEVP & $\left[s\right]$                   &  0.00 & 0.00  & 0.01  & 0.05  & 0.39  & 3.64  & 25.95\\
\hline
\multicolumn{2}{l|}{$c, \tilde{H}^{1/2}$} & 0.333 & 0.333 & 0.333 & 0.333 & 0.333 & 0.333 & 0.333\\
\multicolumn{2}{l|}{$C, \tilde{H}^{1/2}$} & 0.496 & 0.499 & 0.500 & 0.500 & 0.500 & 0.500 & 0.500\\
EVP & $\left[s\right]$                    & 0.00  & 0.00  & 0.01  & 0.03  & 0.17  & 1.20  & 9.48\\
\end{tabular}
}
\end{table}

\section*{Performance} We have seen that lumping decreases construction time of
the norm operator without ruining the properties of the resulting matrix as a
Schur complement preconditioner. The search for an efficient way of solving the
EVP is therefore partially justified. It remains to see how the current
construction times compare with the other computationally demanding tasks
required to solve the coupled multiphysics problem: construction of the
algebraic multigrid preconditioner for $A$ matrix and solution of $Ax=b$ with
preconditioner conjugate gradient method. For the latter we enforce rather
strict convergence criterion - absolute error of $10^{-14}$.

Table \ref{tab:performance} shows comparison of the aforementioned tasks, AMG
and CG, with solution times of GEVP and EVP. We note that the timings of
eigenvalue solvers are obtained with the unoptimized LAPACK routines. Despite
that, the aggregate AMG and CG time is comparable with the EVP. With the exception 
of $n=11$, where $2049$ eigenpairs are obtained, the same holds for GEVP. If we
substitute for GEVP and EVP times the values from table \ref{tab:eig_scale} it
becomes rather evident that the eigenvalue construction of our preconditioner is
\textit{not} a significant bottleneck.
%
\begin{table}[ht]
  \caption{Timings for algebraic multigrid preconditioner of A, AMG preconditioned 
  conjugate gradient solve of $Ax=b$ and solvers of (generalized)eigenvalue
  problems.}
\label{tab:performance}
\footnotesize{
\begin{tabular}{lr|ccccccc}
\multicolumn{2}{l|}{$n=2^i$} &  7    &     8 &     9 & 10    & 11\\
\hline
AMG   & $\left[s\right]$     & 0.03  &  0.05 &  0.22 &  0.87 & 3.50\\
CG    & $\left[s\right]$     & 0.02  &  0.10 &  0.46 &  1.91 & 7.70\\
EVP   & $\left[s\right]$     & 0.01  & 0.03  & 0.17  & 1.20  & 9.48\\
GEVP  & $\left[s\right]$     & 0.01  & 0.05  & 0.39  & 3.64  & 25.95\\
\end{tabular}
}
\end{table}


\end{document}
