\documentclass[10pt, a4paper]{article}
\usepackage{amsmath, amsfonts, amssymb}
\usepackage{enumerate}
\usepackage{caption, subcaption, floatrow}
\usepackage{graphicx}
\usepackage{multirow}
\usepackage{xfrac}
\usepackage{amsmath, amsfonts, amssymb}
\usepackage{tikz}
\usepackage{pgfplots}
\usepackage{pgfplotstable}
\usetikzlibrary{plotmarks, calc}
\usepackage{standalone}
\usepackage{url}

%%%%%%% COLORS
\usepackage{xcolor, colortbl}
\newcolumntype{S}{>{\columncolor{lime!50}}c}         % serial small
\newcolumntype{L}{>{\columncolor{cyan!50}}c}         % serial large
\newcolumntype{P}{>{\columncolor{red!50}}c}          % parallel& 
%%%%%%%

\newcommand{\reals}{\mathbb{R}}
\newcommand{\set}[1]{\{#1\}}
\newcommand{\abs}[1]{\lvert#1\rvert}
\newcommand{\semi}[1]{\lvert#1\rvert}
\newcommand{\norm}[1]{\lVert#1\rVert}
\newcommand{\inner}[2]{\ensuremath{\left(#1, #2\right)}}
\renewcommand{\brack}[1]{\langle#1\rangle}
\newcommand{\average}[1]{\ensuremath{\langle#1\rangle} }
\newcommand{\jump}[1]{\ensuremath{[\![#1]\!]} }
\newcommand{\mat}[1]{\ensuremath{\mathsf{#1}}}
\newcommand{\dual}[1]{\ensuremath{{#1}^{\prime}}}
\renewcommand{\vec}[1]{\mat{#1}}
\newcommand{\supp}{\operatorname{supp}} 

\oddsidemargin=3pt
\textwidth=500pt
\topmargin=3pt

\title{The efficiency of Q-cap and W-cap preconditioners}

\begin{document}
\maketitle

\section*{Introduction}
This note collects new results that address comments of the reviewers on our paper. The comments boil down
to the following four points: (i) The computational results presented are four rather small problems.
(ii) The results are missing cost evaluation. (iii) It would be nice to compare the proposed preconditioners
with some existing alternatives. (iv) It would be nice if the eigenvalue problem of the $Q$-cap were solved
analytically (as much as possible).

\section*{Preliminaries}
The results below are for the most part obtained on a desktop PC with 16 GB of RAM and four Intel\textsuperscript{\copyright} 
Core\texttrademark i5-2500 cpus clocking at 3.3 GHz. The small eigenvalue problems required for constructing the approximations
of fractional Sobolev norms are solved with routines from OpenBlas\cite{openblas}. The large eigenvalue problems that are used
to obtain condition numbers of the $Q$, $W$-cap precondtioned systems are solved with routines from SLEPc\cite{slepc} - in particular,
a generalized Davidson method with Cholesky preconditioning is employed. Finally the implementation of algebraic multigrid (AMG) is
taken from PETSc\cite{petsc}.

In the experiments we shall consider two geometrical configurations. \textit{Uniform} mesh refers to a unit square (2d domain) coupled 
to a horizontal line through $y=\tfrac{1}{2}$ (1d domain). The domain is discretized uniformly into $2N^2$ triangular cells and $N$ 
interval cells. \textit{Non-uniform} mesh has as the two dimensional domain a bi-unit square coupled to a $Y$-shaped bifurcation which
is the one dimensional domain. The discretization of \textit{non-uniform} mesh is unstructured. An important distinction between these
two cases is that in the latter one the number of degrees of freedom on the 1d domain grows faster upon refinement. 

Finally, we shall denote as $n$ the dimension of the space where the 2d unknown is sought while $m$ is the dimension of the spaces of
the 1d uknown (and the Lagrange multiplier). Obviously $n+2m$ is the total size of the linear system to be solved.

\section*{Results}
Condition numbers of the $Q$-cap and $W$-cap preconditioned system are listed in respecively in table \ref{tab:con_Qcap}
and \ref{tab:con_Wcap}. In all four cases they are nicely bounded. Note that in comparison to the
results presented in the paper the size of the considered problems has grown about eight times. Also note that for $\epsilon \ll 1$ the 
condition number is close to that of the Schur complement preconditioner.
%Qcap cond
%Unif
\begin{table}[ht]
  \caption{Condition numbers of the $Q$-cap preconditioned system for different values of parameter 
$\epsilon$. \textit{Uniform} mesh top, \textit{non-uniform} mesh bottom.
}
\label{tab:con_Qcap}
\footnotesize{
\begin{tabular}{l|ccccccc}
\hline
($n$, $m$)\textbackslash $\epsilon$ & $10^{-3}$ & $10^{-2}$ & $10^{-1}$ & $10^{0}$ & $10^{1}$ & $10^{2}$ & $10^{3}$\\
\hline
(25, 5) & 2.633 & 2.760 & 3.686 & 5.716 & 6.427 & 6.514 & 6.523\\
(81, 9) & 2.655 & 2.969 & 4.786 & 6.979 & 7.328 & 7.357 & 7.360\\
(289, 17) & 2.698 & 3.323 & 5.966 & 7.597 & 7.697 & 7.715 & 7.717\\
(1089, 33) & 2.778 & 3.905 & 7.031 & 7.882 & 7.818 & 7.816 & 7.816\\
(4225, 65) & 2.932 & 4.769 & 7.830 & 8.016 & 7.855 & 7.843 & 7.843\\
(16641, 129) & 3.217 & 5.857 & 8.343 & 8.081 & 7.868 & 7.854 & 7.852\\
(66049, 257) & 3.710 & 6.964 & 8.637 & 8.113 & 7.872 & 7.856 & 7.855\\
\hline
\hline
(1466, 55) & 2.773 & 3.811 & 6.401 & 7.181 & 7.205 & 7.207 & 7.208\\
(5584, 107) & 2.926 & 4.597 & 7.018 & 7.314 & 7.282 & 7.280 & 7.279\\
(21905, 212) & 3.181 & 5.453 & 7.381 & 7.365 & 7.308 & 7.304 & 7.303\\
(87260, 422) & 3.630 & 6.374 & 7.689 & 7.493 & 7.426 & 7.421 & 7.421\\
\hline
\end{tabular}
}
\end{table}
%
%Vcap cond
%Unif
\begin{table}[hb]
  \caption{Condition numbers of the $W$-cap preconditioned system for different values of parameter 
$\epsilon$. \textit{Uniform} mesh top, \textit{non-uniform} mesh bottom.
}
\label{tab:con_Wcap_unif}
\footnotesize{
\begin{tabular}{l|ccccccc}
\hline
($n$, $m$)\textbackslash $\epsilon$ & $10^{-3}$ & $10^{-2}$ & $10^{-1}$ & $10^{0}$ & $10^{1}$ & $10^{2}$ & $10^{3}$\\
\hline
(25, 5) & 2.618 & 2.612 & 2.514 & 3.292 & 3.945 & 4.038 & 4.048\\
(81, 9) & 2.619 & 2.627 & 2.546 & 3.615 & 3.998 & 4.044 & 4.048\\
(289, 17) & 2.623 & 2.653 & 2.780 & 3.813 & 4.023 & 4.046 & 4.049\\
(1089, 33) & 2.631 & 2.692 & 3.194 & 3.925 & 4.036 & 4.048 & 4.049\\
(4225, 65) & 2.644 & 2.740 & 3.533 & 3.986 & 4.042 & 4.048 & 4.049\\
(16641, 129) & 2.668 & 2.788 & 3.761 & 4.017 & 4.046 & 4.049 & 4.049\\
(66049, 257) & 2.703 & 3.066 & 3.896 & 4.033 & 4.047 & 4.049 & 4.049\\
\hline
\hline
(1466, 55) & 2.637 & 2.713 & 3.385 & 3.962 & 4.040 & 4.048 & 4.049\\
(5584, 107) & 2.656 & 2.765 & 3.678 & 4.006 & 4.045 & 4.048 & 4.049\\
(21905, 212) & 2.687 & 2.896 & 3.850 & 4.028 & 4.047 & 4.049 & 4.049\\
(87260, 422) & 2.727 & 3.274 & 3.941 & 4.038 & 4.048 & 4.049 & 4.049\\
\hline
\end{tabular}
}
\end{table}

Iteration counts of the preconditioner MinRes method with $Q$, $W$-cap preconditioners are listed in table \ref{tab:iter}. For 
convergence, absolute error of the preconditioned residaul is required to be less than $10^{-12}$. Note that this is stricter 
than in the paper. The iteration counts are nicely bounded
%Note on convergence properties
\footnote{In this example we have also monitored the $H^1_0$ norm of the $2d$ solution. For problems with less than 1mil dofs the
error was interpolated into DG$_4$ space. For larger system the error was computed in the same space as the numerical solution. 
With these metrics the error decreases linearly (or better) until penultimate discretization where it begins to stagnate.}. Note
that as in the paper $W$-cap comes up slightly more efficient for $\epsilon\gg 1$. This was observed in the original results as well.

Note that the largest problems considered in this experiments outgrow those from the paper by approximately factor ten. The size of 
these problems is such that they barely fit into the RAM. For larger problems one shall therefore assemble and solve the systems in 
parellel. At the moment we only have a serial implementation of the trace matrix and so distributed computing is not feasible. However
we remark that the eigenvalue problem required by the $Q$-cap preconditioner is still very small and could be easily computed on a single
cpu and then broadcasted appropriately to the remaining processes.

The new results do not contradict those presented in the paper in any way. Moreover, both the eigenvalue problems and the linear systems
are quite large. Altogether, they should make the reviewers happy.
%Qcap iters
\begin{table}[ht]
  \caption{Iteration counts with the $Q$-cap preconditioned system (top) and $W$-cap preconditioned system (bottom) for different 
values of parameter $\epsilon$. \textit{Uniform} mesh.
}
\label{tab:iter}
\footnotesize{
\begin{tabular}{l|ccccccc}
\hline
($n$, $m$)\textbackslash $\epsilon$ & $10^{-3}$ & $10^{-2}$ & $10^{-1}$ & $10^{0}$ & $10^{1}$ & $10^{2}$ & $10^{3}$\\
\hline
(66049, 257) & 20 & 34 & 37 & 32 & 28 & 24 & 21\\
(263169, 513) & 22 & 34 & 34 & 30 & 26 & 24 & 20\\
(1050625, 1025) & 24 & 33 & 32 & 28 & 26 & 22 & 18\\
(4198401, 2049) & 26 & 32 & 30 & 26 & 24 & 20 & 17\\
(8392609, 2897) & 26 & 30 & 30 & 26 & 22 & 19 & 15\\
(11068929, 3327) & 26 & 30 & 30 & 26 & 22 & 19 & 15\\
\hline
\hline
(66049, 257) & 17 & 33 & 40 & 30 & 20 & 14 & 12\\
(263169, 513) & 19 & 35 & 39 & 28 & 19 & 14 & 11\\
(1050625, 1025) & 22 & 34 & 37 & 27 & 19 & 14 & 11\\
(4198401, 2049) & 24 & 34 & 34 & 25 & 17 & 12 & 9\\
(8392609, 2897) & 25 & 32 & 32 & 24 & 16 & 11 & 8\\
(11068929, 3327) & 25 & 32 & 32 & 25 & 16 & 13 & 11\\
\hline
\end{tabular}
}
\end{table}

\subsection*{Computation Costs} We shall now address the second point of the reviewers' comments. In particular, we will focus 
on the setup cost of both preconditioners. In case of the $Q$-cap preconditioner the operator to be assembled takes the form
$\text{diag}(AMG(A_2), A^{-1}, H^{-1})$, where $A_2$, $A$ are the discretized 2d, 1d Laplacians and $H$ is the matrix assembled from
the eigenvalue problem $Ax=\lambda M x$ with $M$ the one dimensional mass matrix. The $W$-cap preconditioner is then 
$\text{diag}(AMG(A_2+\transp{T}AT)), A^{-1}, C)$ where $T$ is the 2d-1d trace/restriction and matrix $C=M^{-1}AM^{-1}$. 

%Rates



\section*{Efficiency}

\section*{Todo}
Address points (iii) and (iv).

\bibliographystyle{plain}
\bibliography{report}

\end{document}
