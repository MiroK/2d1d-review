\documentclass[10pt, a4paper]{article}
\usepackage{amsmath, amsfonts, amssymb}
\usepackage{enumerate}
\usepackage{caption, subcaption, floatrow}
\usepackage{graphicx}
\usepackage{multirow}
\usepackage{xfrac}
\usepackage{amsmath, amsfonts, amssymb}
\usepackage{tikz}
\usepackage{pgfplots}
\usepackage{pgfplotstable}
\usetikzlibrary{plotmarks, calc}
\usepackage{standalone}
\usepackage{url}

%%%%%%% COLORS
\usepackage{xcolor, colortbl}
\newcolumntype{S}{>{\columncolor{lime!50}}c}         % serial small
\newcolumntype{L}{>{\columncolor{cyan!50}}c}         % serial large
\newcolumntype{P}{>{\columncolor{red!50}}c}          % parallel& 
%%%%%%%

\newcommand{\reals}{\mathbb{R}}
\newcommand{\set}[1]{\{#1\}}
\newcommand{\abs}[1]{\lvert#1\rvert}
\newcommand{\semi}[1]{\lvert#1\rvert}
\newcommand{\norm}[1]{\lVert#1\rVert}
\newcommand{\inner}[2]{\ensuremath{\left(#1, #2\right)}}
\renewcommand{\brack}[1]{\langle#1\rangle}
\newcommand{\average}[1]{\ensuremath{\langle#1\rangle} }
\newcommand{\jump}[1]{\ensuremath{[\![#1]\!]} }
\newcommand{\mat}[1]{\ensuremath{\mathsf{#1}}}
\newcommand{\dual}[1]{\ensuremath{{#1}^{\prime}}}
\renewcommand{\vec}[1]{\mat{#1}}
\newcommand{\supp}{\operatorname{supp}} 

\oddsidemargin=3pt
\textwidth=500pt
\topmargin=3pt

\title{Efficiency of Q-cap and W-cap preconditioners}

\begin{document}
\maketitle

\section*{Introduction}
This note collects new results that address comments of the reviewers on our paper. 
The comments boil down to the following four points: (i) The computational results 
presented are four rather small problems. (ii) The results are missing cost evaluation. 
(iii) It would be nice to compare the proposed preconditioners with some existing 
alternatives. (iv) It would be nice if the eigenvalue problem of the $Q$-cap were solved
analytically (as much as possible).

\section*{Preliminaries}
The results below are for the most part obtained on a desktop PC with 16 GB of RAM 
and four Intel Core i5-2500 cpus clocking at 3.3 GHz. The small eigenvalue problems 
required for constructing the approximations of fractional Sobolev norms are solved 
with routines from LAPACK\cite{lapack} implemented in OpenBlas\cite{openblas}. The 
large eigenvalue problems that are used to obtain condition numbers of the $Q$, $W$-cap 
precondtioned systems are solved with routines from SLEPc\cite{slepc} - in particular,
a generalized Davidson method with Cholesky preconditioning is employed. Finally the 
implementation of algebraic multigrid (AMG) is taken from PETSc\cite{petsc}.

In the experiments we shall consider two geometrical configurations. \textit{Uniform} 
mesh refers to a unit square (2$d$ domain) coupled to a horizontal line through 
$y=\sfrac{1}{2}$ (1$d$ domain). The domain is discretized uniformly into $2N^2$ 
triangular cells and $N$ interval cells. \textit{Non-uniform} mesh has as the two 
dimensional domain a bi-unit square coupled to a $Y$-shaped bifurcation which
is the one dimensional domain. The discretization of \textit{non-uniform} mesh is 
unstructured. An important distinction between these two cases is that for
comparable numbers of degrees of freedom(dofs) on a 1$d$ domain the
\textit{uniform} mesh has more dofs in 2$d$ then the \textit{non-uniform} one.
Hoever, the latter one represents a less academical configuration.

Finally, we shall denote as $n$ the dimension of the space where the 2$d$ unknown 
is sought while $m$ is the dimension of the spaces of the 1$d$ uknown (and the 
Lagrange multiplier). Obviously $n+2m$ is the total size of the linear system to 
be solved.

\section*{Results}
Condition numbers of the $Q$-cap and $W$-cap preconditioned system are listed in
respecively in tables \ref{tab:cond_Qcap} and \ref{tab:cond_Wcap}. In all four 
cases they are nicely bounded. Note that in comparison to the results presented 
in the paper the size of the considered problems has grown about eight times. Also 
note that for $\epsilon \ll 1$ the condition number is close to that of the Schur 
complement preconditioner (as was true in the paper).
%Qcap cond
%Unif
\begin{table}[ht]
  \caption{Condition numbers of the $Q$-cap preconditioned system for different 
  values of parameter $\epsilon$. \textit{Uniform} mesh top, \textit{non-uniform} 
  mesh bottom.
}
\label{tab:cond_Qcap}
\footnotesize{
\begin{tabular}{l|ccccccc}
\hline
($n$, $m$)\textbackslash $\epsilon$ & $10^{-3}$ & $10^{-2}$ & $10^{-1}$ & $10^{0}$ & $10^{1}$ & $10^{2}$ & $10^{3}$\\
\hline
(25, 5) & 2.633 & 2.760 & 3.686 & 5.716 & 6.427 & 6.514 & 6.523\\
(81, 9) & 2.655 & 2.969 & 4.786 & 6.979 & 7.328 & 7.357 & 7.360\\
(289, 17) & 2.698 & 3.323 & 5.966 & 7.597 & 7.697 & 7.715 & 7.717\\
(1089, 33) & 2.778 & 3.905 & 7.031 & 7.882 & 7.818 & 7.816 & 7.816\\
(4225, 65) & 2.932 & 4.769 & 7.830 & 8.016 & 7.855 & 7.843 & 7.843\\
(16641, 129) & 3.217 & 5.857 & 8.343 & 8.081 & 7.868 & 7.854 & 7.852\\
(66049, 257) & 3.710 & 6.964 & 8.637 & 8.113 & 7.872 & 7.856 & 7.855\\
\hline
\hline
(1466, 55) & 2.773 & 3.811 & 6.401 & 7.181 & 7.205 & 7.207 & 7.208\\
(5584, 107) & 2.926 & 4.597 & 7.018 & 7.314 & 7.282 & 7.280 & 7.279\\
(21905, 212) & 3.181 & 5.453 & 7.381 & 7.365 & 7.308 & 7.304 & 7.303\\
(87260, 422) & 3.630 & 6.374 & 7.689 & 7.493 & 7.426 & 7.421 & 7.421\\
\hline
\end{tabular}
}
\end{table}
%
%Vcap cond
%Unif
\begin{table}[hb]
  \caption{Condition numbers of the $W$-cap preconditioned system for different 
  values of parameter $\epsilon$. \textit{Uniform} mesh top, \textit{non-uniform} 
  mesh bottom.
}
\label{tab:cond_Wcap}
\footnotesize{
\begin{tabular}{l|ccccccc}
\hline
($n$, $m$)\textbackslash $\epsilon$ & $10^{-3}$ & $10^{-2}$ & $10^{-1}$ & $10^{0}$ & $10^{1}$ & $10^{2}$ & $10^{3}$\\
\hline
(25, 5) & 2.618 & 2.612 & 2.514 & 3.292 & 3.945 & 4.038 & 4.048\\
(81, 9) & 2.619 & 2.627 & 2.546 & 3.615 & 3.998 & 4.044 & 4.048\\
(289, 17) & 2.623 & 2.653 & 2.780 & 3.813 & 4.023 & 4.046 & 4.049\\
(1089, 33) & 2.631 & 2.692 & 3.194 & 3.925 & 4.036 & 4.048 & 4.049\\
(4225, 65) & 2.644 & 2.740 & 3.533 & 3.986 & 4.042 & 4.048 & 4.049\\
(16641, 129) & 2.668 & 2.788 & 3.761 & 4.017 & 4.046 & 4.049 & 4.049\\
(66049, 257) & 2.703 & 3.066 & 3.896 & 4.033 & 4.047 & 4.049 & 4.049\\
\hline
\hline
(1466, 55) & 2.637 & 2.713 & 3.385 & 3.962 & 4.040 & 4.048 & 4.049\\
(5584, 107) & 2.656 & 2.765 & 3.678 & 4.006 & 4.045 & 4.048 & 4.049\\
(21905, 212) & 2.687 & 2.896 & 3.850 & 4.028 & 4.047 & 4.049 & 4.049\\
(87260, 422) & 2.727 & 3.274 & 3.941 & 4.038 & 4.048 & 4.049 & 4.049\\
\hline
\end{tabular}
}
\end{table}

Iteration counts of the preconditioner MinRes method with $Q$, $W$-cap preconditioners 
are listed in table \ref{tab:iter}. For convergence, absolute error of the preconditioned 
residaul is required to be less than $10^{-12}$. Note that this is stricter 
than in the paper. The iteration counts are nicely bounded
%Note on convergence properties
\footnote{In this example we have also monitored the $H^1_0$ norm of the 2$d$ solution. 
For problems with less than one million dofs the error was interpolated into DG$_4$ 
space. For larger system the error was computed in the same space as the numerical 
solution. With these metrics the error decreases linearly (or better) until penultimate 
discretization where it begins to stagnate.}. Note
that as in the paper $W$-cap comes out slightly more efficient for $\epsilon\gg 1$. 
This was observed in the original results as well.

Note that the largest problems considered in this experiments outgrow those from 
the paper by approximately factor ten. These problems are such that they barely 
fit into the RAM. For even larger problems one shall therefore assemble and solve 
the systems in parellel. At the moment we only have a serial implementation of the 
trace matrix and so distributed computing is not feasible. However we remark that 
the eigenvalue problem required by the $Q$-cap preconditioner in the examples is 
still very small and could be easily computed on a single cpu and then broadcasted 
appropriately to the remaining processes.

The new results do not contradict those presented in the paper in any way. Moreover, 
both the eigenvalue problems and the linear systems are quite large. Altogether, 
they should make the reviewers happy.
%Qcap iters
\begin{table}[ht]
  \caption{Iteration counts with the $Q$-cap preconditioned system (top) and $W$-cap 
  preconditioned system (bottom) for different values of parameter $\epsilon$. 
  \textit{Uniform} mesh.
}
\label{tab:iter}
\footnotesize{
\begin{tabular}{l|ccccccc}
\hline
($n$, $m$)\textbackslash $\epsilon$ & $10^{-3}$ & $10^{-2}$ & $10^{-1}$ & $10^{0}$ & $10^{1}$ & $10^{2}$ & $10^{3}$\\
\hline
(66049, 257) & 20 & 34 & 37 & 32 & 28 & 24 & 21\\
(263169, 513) & 22 & 34 & 34 & 30 & 26 & 24 & 20\\
(1050625, 1025) & 24 & 33 & 32 & 28 & 26 & 22 & 18\\
(4198401, 2049) & 26 & 32 & 30 & 26 & 24 & 20 & 17\\
(8392609, 2897) & 26 & 30 & 30 & 26 & 22 & 19 & 15\\
(11068929, 3327) & 26 & 30 & 30 & 26 & 22 & 19 & 15\\
\hline
\hline
(66049, 257) & 17 & 33 & 40 & 30 & 20 & 14 & 12\\
(263169, 513) & 19 & 35 & 39 & 28 & 19 & 14 & 11\\
(1050625, 1025) & 22 & 34 & 37 & 27 & 19 & 14 & 11\\
(4198401, 2049) & 24 & 34 & 34 & 25 & 17 & 12 & 9\\
(8392609, 2897) & 25 & 32 & 32 & 24 & 16 & 11 & 8\\
(11068929, 3327) & 25 & 32 & 32 & 25 & 16 & 13 & 11\\
\hline
\end{tabular}
}
\end{table}

\subsection*{Computation Costs} We shall now address the second point of the 
reviewers' comments. In particular, we will focus on the setup cost of both 
preconditioners. For simplicity we let $\epsilon=1$. In case of the $Q$-cap 
preconditioner the operator to be assembled takes the form $\text{diag}(AMG(A_2), A^{-1}, H^{-1})$, 
where $A_2$, $A$ are the discretized 2$d$, 1$d$ Laplacians and $H$ is the matrix assembled 
from the eigenvalue problem $Ax=\lambda M x$ with $M$ the one dimensional mass matrix. 
The $W$-cap preconditioner is then $\text{diag}(AMG(A_2+\dual{T}AT)), A^{-1}, C)$ 
where $T$ is the 2$d$-1$d$ trace/restriction and matrix $C=M^{-1}AM^{-1}$. 

Clearly, the most costly operations required by the $Q$-cap preconditioner are
computation of AMG and solving the generalized eigenvalue problem(GEVP). Unlike
$Q$-cap, the $W$-cap precondtioner requires AMG on a new operator
$A_2+\dual{T}AT$, which is not computed as part of the system(left hand side). 
For completeness we will monitor costs of the assembly (ADD). We remark that in our
implementation the product term is not assembled with the dedicated PETSc
routine \texttt{MatPtAP} but with two matrix-matrix multiplications. This is
potentially suboptimal. Finally we shall keep track of the total cpu time
required for convergence of preconditioned MinRes.

\begin{table}[ht]
  \caption{Timings of elements of construction of the $Q$, $W$-cap
  preconditioners on \textit{uniform} mesh. The numbers in the brackets show 
  estimated complexity of computing quantitiy $v$ at $i$-th row,
  $r_i=\sfrac{\log{v_i}-\log{v_{i-1}}}{\log{m_i}-\log{m_{i-1}}}$. Final row is
  the least squares fit of the reported data(the last row of AMG, ADD
  timings is ignored) giving complexity $v=\mathcal{O}(m^r)$.
}
\label{tab:timings_unif}
\footnotesize{
\begin{tabular}{c|cccccc}
\hline
 & \multicolumn{2}{c|}{$Q$-cap} & \multicolumn{2}{c}{$W$-cap}\\
\hline
  $m$ & AMG$\left[s\right]$ & GEVP$\left[s\right]$ & MinRes$\left[s\right]$ &
        AMG$\left[s\right]$ & ADD$\left[s\right]$  & MinRes$\left[s\right]$ \\
\hline
  257  & 0.075(1.98)  & 0.014(1.81)  &  0.579(1.69)  & 0.078(1.94)  & 0.012(1.85)   & 0.514(1.73)  \\
  513  & 0.299(2.01)  & 0.066(2.27)  &  2.286(1.99)  & 0.309(1.99)  & 0.047(1.94)   & 2.019(1.98)  \\
  1025 & 1.201(2.01)  & 0.477(2.87)  &  8.032(1.82)  & 1.228(1.99)  & 0.185(1.99)   & 7.909(1.97)  \\
  2049 & 4.983(2.05)  & 3.311(2.80)  &  30.814(1.94) & 4.930(2.01)  & 0.738(2.00)   & 30.309(1.94) \\
  2897 & 9.686(1.92)  & 8.384(2.68)  &  62.666(2.05) & 10.644(2.22) & 1.463(1.98)   & 59.132(1.93) \\
  3327 & 15.940(3.60) & 12.253(2.74) &  84.431(2.15) & 15.650(2.79) & 11.171(14.69) & 82.132(2.37) \\
\hline
% Qamg 2.02 Qeig 2.70 Vamg 2.02 Vadd 1.98 3761 8886110
  & (2.02) & (2.70) & (1.92) &
    (2.02) & (1.98) & (1.96) \\
\hline
\end{tabular}
}
\end{table}
%
The observed timings are reported in tables \ref{tab:timings_unif},
\ref{tab:timings_nonunif}. We remark that with both \textit{uniform} and \textit{non-uniform} 
mesh the numbers reported for AMG and ADD on the final discretizations deviate 
from the trend set by the predecessors. This is likely because SWAP memory was 
required to complete the operations and the cases should therefore be omitted 
from the discussion. In the remaining cases we observe that AMG construction 
scales quadratically with $m$ and the inclussion of the trace term has little effect 
on the construction time. Moreover assembly of $A_2+\dual{T}AT$ has negligible 
costs in comparison to AMG.

For smaller problem the $W$-cap preconditioner is about as expansive to construct 
as the $Q$-cap preconditioner. For larger problems, requiring order of thousand
eigenvalues, the contribution of GEVP solve time to the total cost becomes
evident. As a result $Q$-cap becomes about twice as expensive to construct as the 
$W$-cap. We remark that the solution times with both preconditioners are
practically identical. Moreover the solution time dominates the construction
time.

In our results, GEVP takes always less time than AMG. However, based on mesh, 
the employed solver, \texttt{LAPACK.SYGVD}, appears to be $\mathcal{O}(m^{2.7})$
or $\mathcal{O}(m^{2.8})$ in complexity and thus with increasing $m$, solving 
the eigenvalue problem might dominate the construction of the $Q$-cap preconditioner.
Using the least squares fit\footnote{Note that AMG, ADD do not include the last
discretizations effected by the swapping into considerations. For GEVP all the
reported timings are included.} we can estimate the size of the problem where the
cost of GEVP matches that of AMG. On the \textit{uniform} mesh, that moment is
projected to happen at $(n, m)=(8886110, 3761)$. On the other hand, the
unstructured discretization results in $(24154952, 7527)$. 

Based on the observed timings, we conclude that $W$-cap preconditioner is more
practical than the $Q$-cap preconditioner. However, $Q$-cap preconditioner is
certainly \textit{not} impractical. One can of course imagine conditions where
the size of the one dimensional problem grows faster than reported here, e.g.
mesh refined heavily towards the 1$d$ domain, or, 1$d$ domain being a space
filling curve, but for many applications the $Q$-cap can be an efficient 
preconditioner.
%
\begin{table}[ht]
  \caption{Timings of elements of construction of the $Q$, $W$-cap
  preconditioners on \textit{non-uniform} mesh. The numbers in the brackets show 
  estimated complexity of computing quantitiy $v$ at $i$-th row,
  $r_i=\sfrac{\log{v_i}-\log{v_{i-1}}}{\log{m_i}-\log{m_{i-1}}}$. Final row is
  the least squares fit of the reported data(the last row of AMG, ADD
  timings is ignored) giving complexity $v=\mathcal{O}(m^r)$.}
\label{tab:timings_nonunif}
\footnotesize{
\begin{tabular}{c|cccc}
\hline
 & \multicolumn{2}{c|}{$Q$-cap} & \multicolumn{2}{c}{$W$-cap}\\
\hline
  $m$ & AMG$\left[s\right]$ & GEVP$\left[s\right]$ & AMG$\left[s\right]$ & ADD$\left[s\right]$ \\
\hline
212 & 0.046(2.03) & 0.005(1.62) & 0.049(2.10) & 0.005(1.86)\\
422 & 0.188(2.04) & 0.018(2.05) & 0.189(1.96) & 0.018(1.75)\\
841 & 0.783(2.07) & 0.099(2.43) & 0.785(2.07) & 0.066(1.92)\\
1681 & 3.191(2.03) & 1.001(3.35) & 3.190(2.02) & 0.257(1.96)\\
3360 & 13.004(2.03) & 8.004(3.00) & 13.220(2.05) & 0.986(1.94)\\
4514 & 25.465(2.28) & 18.824(2.90) & 30.406(2.82) & 2.178(2.69)\\
\hline
% Qamg 2.04 Qeig 2.80 Vamg 2.03 Vadd 1.90 7527 24154952
  &  (2.04) & (2.80) & (2.03) &  (1.90)\\
\hline
\end{tabular}
}
\end{table}

% Only symmetry - what about tridiagonaly - there are n^2, n*log(n) algorithm
In the previous results the generalized eigenvalue problem was solved with an
algorithm of nearly cubic complexity which only took into account the symmetry of 
matrices of $A$, $M$. However, as we use CG$_1$ discretization, the matrices
are also tridiagonal and the $Q$-cap construction could be made more efficient
with an algorithm which would exploit fully the structure of the eigenvalue problem.
An $\mathcal{O}(m^2)$\footnote{This scaling is established for rather small
problems $m<1000$.}. of this kind is given in \cite{gevp_s3d}. Unfortunatelly it
is not implemented in available linear packages and we do not have resources to
implement it ourselves. Therefore we only comment that \cite{gevp_s3d} is potentially 
a mean to make GEVP scale like AMG.

On the other hand there are routines already implemented in LAPACK for dealing
with eigenvalue problems (EVP) with symmetric tridiagonal matrices, i.e.
$Ax=\lambda x, A=\dual{A}$, which enjoy the quadratic scaling. Therefore, a
possible path to a faster $Q$-cap preconditioner is to transform GEVP into EVP 
with a symmetric tridiagonal matrix. Naturally, such a transformation is only
meaningful if the resulting modified $Q$-cap preconditioner remains a good
preconditioner for our coupled problem.

\section*{Efficiency}\label{sec:eff} Let us first assume that were are able to
transform GEVP and see how much performance can be gained this way. Table
\ref{tab:evp} shows cpu time required to computed $m$ sized eigenvalue problem
$Ax=\lambda x$ in comparison to $m$ sized generalized eigenvalue problem
$Ax=\lambda M x$. Both system are assembled on a \textit{uniform} mesh. In the
later case we use, as before, LAPACK's \texttt{SYGVD} routine while the EVP is
solved with \texttt{LAPACK.STEGR} routine which implements the method of
Multiple Relatively Robust Representations \cite{mmmr} - a nearly
$\mathcal{O}(m^2)$ algorithm. We observe that the method scales as $m^{2.34}$
which is close to what \cite{demmel} reported for a jungle of matrices with size
$m<10^4$. It is evident that there is a lot to gain in efficiency by considering
possible transformation of GEVP.
% Comment on complexity of tridiag. Show the numbers GEVP vs. EVP tridiag.
\begin{table}[ht]
  \caption{Timings of EVP and GEVP solution times solved respectively with
  LAPACK's \texttt{STEGR} and \texttt{SYGVD} routines. The results are obtained
  with Intel Core i5-4570S cpu clocking at 2.9 Ghz.}
\label{tab:evp}
\footnotesize{
\begin{tabular}{c|cc}
\hline
  m & EVP$\left[s\right]$ & GEVP$\left[s\right]$\\
\hline
1025 & 0.080(1.90) & 0.396(2.56)\\
2049 & 0.312(1.96) & 2.631(2.73)\\
4097 & 1.363(2.13) & 18.846(2.84)\\
8193 & 7.717(2.50) & 122.875(2.71)\\
16385 & 53.001(2.78) & 876.484(2.83)\\
\hline
    & (2.34) & (2.78)\\
\hline
\end{tabular}
}
\end{table}

% Define the approximations
Matrix $H$ which is the ultimate diagonal block in the matrix representation of
the $Q$-cap preconditioner is based on eigenpairs $(\lambda, u)$ such that 
$Au=\lambda M u$. In particular, matrices $H_s=M(U\Lambda^s \dual{U})M$, where
$AU=MU\Lambda$, are used in its definition. Let us define $M_l$ as the lumped
mass matrix and cosquently matrices $U_l, \Lambda_l$ which are such that
$AU_l=M_l U_l\Lambda$. Then we shall define an approximation of $H_s$ as
$\hat{H}_s=M(U_l\Lambda^s_l\dual{U})M$. Substituting for $H_s$ in the definition
of $H$ and the $Q$-cap we obtain a modified operator which shall be termed 
$\hat{Q}$-cap preconditioner.

% On computation
Note that $\Lambda_l, U_l$ are GEVP transformed into an EVP with with symmetric 
tridiagonal matrix
\begin{equation}\label{eq:foo}
  M_l^{-1/2}A M_l^{-1/2} V=V\Lambda_l,\quad U_l=M_l^{-1/2} V.
\end{equation}
We ramark that that the \eqref{eq:foo} and the original GEVP have the same
eigenvalues but the eigenvectors must be transformed according to
\eqref{eq:foo}.

Performance of the approximation is summarized in table \ref{tab:hatQcap} which
reports condition numbers of the preconditioned system as well as the iteration
counts of the MinRes method. The experiments are performed on \textit{uniform}
mesh. 

\begin{table}[hb]
  \caption{$\hat{Q}$-cap preconditioned system for different values of parameter $\epsilon$. 
  Condition numbers top half, iteration counts bottom half. \textit{Uniform} mesh.
}
\label{tab:hatQcap}
\footnotesize{
\begin{tabular}{l|ccccccc}
\hline
($n$, $m$)\textbackslash $\epsilon$ & $10^{-3}$ & $10^{-2}$ & $10^{-1}$ & $10^{0}$ & $10^{1}$ & $10^{2}$ & $10^{3}$\\
\hline
(25, 5) & 2.626 & 2.697 & 3.216 & 4.357 & 4.764 & 4.817 & 4.823\\
(81, 9) & 2.636 & 2.788 & 3.677 & 4.704 & 4.846 & 4.872 & 4.875\\
(289, 17) & 2.655 & 2.951 & 4.208 & 4.904 & 4.923 & 4.930 & 4.930\\
(1089, 33) & 2.693 & 3.228 & 4.707 & 5.015 & 4.950 & 4.946 & 4.946\\
(4225, 65) & 2.766 & 3.645 & 5.092 & 5.075 & 4.960 & 4.953 & 4.952\\
(16641, 129) & 2.901 & 4.172 & 5.343 & 5.106 & 4.964 & 4.955 & 4.954\\
(66049, 257) & 3.138 & 4.708 & 5.490 & 5.121 & 4.966 & 4.956 & 4.955\\
\hline
\hline
\end{tabular}
}
\end{table}

% Previous results suggest that the price for lumping seems to be doubling the
% iteration count. This translates to doubling the solution time in comparison to
% the $Q$-cap preconditioner. Note that there the setup cost, which here is
% significanly lower due to efficient solution of transfomed evp.
% \begin{table}[hb]
%   \caption{Condition numbers of the $\hat{Q}$-cap preconditioned system for different 
%   values of parameter $\epsilon$. \texit{Non-uniform} mesh. The condition
%   numbers are unbounded.}
% \label{tab:hatQcap_boom}
% \footnotesize{
% \begin{tabular}{l|ccccccc}
% \hline
% ($n$, $m$)\textbackslash $\epsilon$ & $10^{-3}$ & $10^{-2}$ & $10^{-1}$ & $10^{0}$ & $10^{1}$ & $10^{2}$ & $10^{3}$\\
% \hline
% (1466, 55) & 73.452 & 62.824 & 35.083 & 19.118 & 16.043 & 15.692 & 15.657\\
% (5584, 107) & 196.413 & 150.327 & 71.371 & 37.053 & 30.877 & 30.181 & 30.110\\
% (21905, 212) & 536.752 & 351.690 & 146.398 & 73.915 & 61.369 & 59.961 & 59.818\\
% (87260, 422) & 1416.460 & 776.714 & 296.494 & 147.547 & 122.278 & 119.449 & 119.162\\
% \hline
% \end{tabular}
% }
% \end{table}


% But ununif. Which goes away? Comparison of the spectra. Why?

% The other approximation. Wht

% Why

\section*{Todo}
Address points (iii) and (iv). Some questions: (1) Do we include \S\ref{sec:eff}
into the paper? (2) Follow up to (1) - should we consider other approximations,
e.g. include $M_l$ into $H_s$. In any case, do we also explain why the results
are the way they are? (3) Should all the computations be performed on both meshes
and the results included in the response to reviewers. Note that on
\textit{non-uniform} we do not have an analutical solution to test against.

\bibliographystyle{plain}
\bibliography{report}

\end{document}
