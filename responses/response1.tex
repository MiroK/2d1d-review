\documentclass[10pt, a4paper]{article}
\usepackage{amsmath, amsfonts, amssymb}
\usepackage{enumerate}
\usepackage{caption, subcaption, floatrow, subfig}
\usepackage{graphicx}
\usepackage{multirow}
\usepackage{xfrac}
\usepackage{amsmath, amsfonts, amssymb}
\usepackage{tikz}
\usepackage{pgfplots}
\usepackage{pgfplotstable}
\usetikzlibrary{plotmarks, calc}
\usepackage{standalone}
\usepackage{url}

% ---Shortcuts---
\newcommand*{\Scale}[2][4]{\scalebox{#1}{$#2$}}%
\newcommand*{\Resize}[2]{\resizebox{#1}{!}{$#2$}}%
\newcommand{\reals}{\mathbb{R}}
\newcommand{\naturals}{\mathbb{N}}
\newcommand{\none}{\Scale[1.0]{-1}}
\newcommand{\half}{\Scale[0.5]{\tfrac{1}{2}}}
\newcommand{\nhalf}{\Scale[0.5]{-\tfrac{1}{2}}}
\newcommand{\thalf}{\Scale[0.5]{\tfrac{3}{2}}}
\renewcommand{\div}{\nabla\cdot}
\newcommand{\set}[1]{\{#1\}}
\newcommand{\seq}[1]{\{#1\}}
\newcommand{\abs}[1]{\lvert#1\rvert}
\newcommand{\semi}[1]{\lvert#1\rvert}
\newcommand{\norm}[1]{\lVert#1\rVert}
\renewcommand{\brack}[1]{\langle#1\rangle}
\newcommand{\average}[1]{\ensuremath{\langle#1\rangle} }
\newcommand{\jump}[1]{\ensuremath{[\![#1]\!]} }
% ---Inner products---
\newcommand{\inner}[2]{\ensuremath{\left(#1, #2\right)}}
\newcommand{\HinnerO}[2]{\ensuremath{\left(#1, #2\right)_{1, \Omega}}}
\newcommand{\HinnerG}[2]{\ensuremath{\left(#1, #2\right)_{1, \Gamma}}}
\newcommand{\innerO}[2]{\ensuremath{\left(#1, #2\right)_{0, \Omega}}}
\newcommand{\innerG}[2]{\ensuremath{\left(#1, #2\right)_{0, \Gamma}}}
% ---Norms---
\newcommand{\normO}[1]{\lVert#1\rVert_{0, \Omega}}
\newcommand{\normG}[1]{\lVert#1\rVert_{0, \Gamma}}
\newcommand{\HnormO}[1]{\lvert#1\rvert_{1, \Omega}}
\newcommand{\HnormG}[1]{\lvert#1\rvert_{1, \Gamma}}
\newcommand{\hnorm}[1]{\lVert#1\rVert_{\half}}
\newcommand{\nhnorm}[1]{\lVert#1\rVert_{\nhalf}}
% ---Spaces---
\newcommand{\HO}{H^{1}_0\left(\Omega\right)}
\newcommand{\HG}{H^{1}_0\left(\Gamma\right)}
\newcommand{\Dmo}{H^{-1}_0\left(\Gamma\right)} % dual - 1 ONE
\newcommand{\Dmh}{H^{-\half}\left(\Gamma\right)}
\newcommand{\HGO}{H^{\half}_{00}\left(\Gamma\right)}
% ---Intervals---
\newcommand{\open}[1]{\left(#1\right)}
\newcommand{\close}[1]{\left[#1\right]}
% ---Linalg---
\newcommand{\Amat}{\ensuremath{\mathsf{A}}}
\newcommand{\Bmat}{\ensuremath{\mathsf{B}}}
\newcommand{\Mmat}{\ensuremath{\mathsf{M}}}
\newcommand{\Tmat}{\ensuremath{\mathsf{T}}}
\newcommand{\Hmat}[1]{\ensuremath{\mathsf{H}\!\left(#1\right)}}
\newcommand{\mat}[1]{\ensuremath{\mathsf{#1}}}
\newcommand{\inv}[1]{\ensuremath{{#1}^{-1}}}
\newcommand{\ninv}[1]{\ensuremath{{#1}^{\Scale[0.5]{-1}}}}
\newcommand{\transp}[1]{\ensuremath{{#1}^{\Scale[0.5]{\top}}}}
\renewcommand{\vec}[1]{\mat{#1}}
\newcommand{\spn}{\operatorname{span}\ }
% Short for math cal
\newcommand{\op}[1]{\ensuremath{\mathcal{#1}}}
% Stuff where we might want to change notation
\newcommand{\AU}{\ensuremath{A_U}}  % What was once A_1
\newcommand{\AV}{\ensuremath{A_V}}  %               A_2
\newcommand{\BU}{\ensuremath{B_U}}  %               B_1
\newcommand{\BV}{\ensuremath{B_V}}  %               B_2
\newcommand{\BBW}{\ensuremath{\mathcal{B}_{W}}}  %  W cap preconditioner
\newcommand{\BBQ}{\ensuremath{\mathcal{B}_{Q}}}  %  Q cap preconditioner
% Matrix representation of the system operator, its components and the
% preconditioner
\newcommand{\AAh}{\ensuremath{\mathbb{A}}}
\newcommand{\BBh}{\ensuremath{\mathbb{B}}}
\newcommand{\BBWh}{\ensuremath{\mathbb{B}_{W}}}
\newcommand{\BBQh}{\ensuremath{\mathbb{B}_{Q}}}
% ---Change inf and sup alignments---
\makeatletter
\renewcommand{\inf}{\mathop{\@inf\vphantom{\@sup}}}
\renewcommand{\sup}{\mathop{\@sup\vphantom{\@inf}}}
\newcommand{\@inf}{\operatorname*{inf}}
\newcommand{\@sup}{\operatorname*{sup}}
\newcommand{\overbar}[1]{\mkern 1.5mu\overline{\mkern-1.5mu#1\mkern-1.5mu}\mkern 1.5mu}

\oddsidemargin=3pt
\textwidth=500pt
\topmargin=3pt

\title{\large{Response to the reviewer}}
\date{}

\begin{document}
\maketitle

We would like to thank the reviewer for his/hers helpful comments on our manuscript. 
Changes made to the paper that reflect them are summarized below. The main
comments are addressed as follows:

%---------------------- MAJOR 
\begin{enumerate}
%%%%
  \item{\textit{As pointed out in the article, the $Q$-cap preconditioner
    requires for its realization the eigenvalues of the stiffness matrix. How
    does this point influence the practicability of the approach? Moreover how
    expensive is the solution of the eigenvalue problem in comparison to the
    application of the preconditioner? Also, some comparison results of the two
    approaches with respect cot time/costs would be helpful. The article would
    benefit if some additional informations about the efficiency of the proposed
    methods would be given.
    }
}
%%%%
  \item{\textit{That about alternative approaches for the realising norms in
    fractional Sobolev spaces? The authors may also comment on other approaches
    such as the BPX-approach, multigrid methods on the boundary or discrete
    boundary integral operators.}
    }
%%%%
  \item{\textit{The weak formulation (2.11) is not consistent with (1.1),
    particularly the $\epsilon$ in the term $\brack{p, \epsilon T_\Gamma
    \phi}_{\Gamma}$. Please correct this. Note, if we would start with (1.1) the
    formulation is non-symmetric and so is (2.12).}
}
%%%%
  \item{\textit{At the end of Section 3.2, the authors present a necessary
    condition for the inf-sup stability. Maybe a comment for the case of
    piecewise constant approximations for $Q_h$ can be given.}
}
%%%%
  \item{\textit{The proof of Lemma 3.4 is evident and well known. Stating the
    inf-sup condition and a citation for the proof is okay.}
}
%%%%
  \item{\textit{In Table 3.2 and Table 4., the authors presented numerical
    results including iteration numbers. They tend to increase and decrease for
    the values of $\epsilon$ from $10^{-3}$ to $10^{3}$. Do the authors have an
    explanation for this behavior?}
}
%%%%
  \item{\textit{Numerical examples for the 3d case, which are application driven
    (e.g. from porous media), may lead to a great improvement of the article.
    This is not essential for the manuscript.}
}
\end{enumerate}

The following are the changes due to minor comments:
%---------------------- MINOR 

\begin{enumerate}
\setcounter{enumi}{7}
%%%%
  \item{\textit{First sentence of the proof of Lemma 3.2 a full stop is missing.}\\
    This has been fixed.
}
%%%%
  \item{\textit{In the estimate after (3.13) on page 11 there should be $p_2$
    instead of $p_1$.}\\
    The inequality has been changed to read $\norm{v_h}_{0, \Gamma}\leq
    \norm{p_2}_{0, \Gamma}$.
}
%%%%
  \item{\textit{In Table 3.2 the authors write ``...of size $n_Q=9409$.'' In the
    table numbers for $n_Q$ are listed up to 97, which is not clear at the first
    sight. Please be more precise.}
}
%%%%
  \item{\textit{For the results in Table 3.2 and Table 4.2 also the number of
    total degrees of freedom would be helpful.}
}
%%%%
  \item{\textit{On page 15, the constant $c_1$ in the first inequality is placed
    wrong.}
}
%%%%
  \item{\textit{On page 15, correct the inequality (3.21).}
}
%%%%
  \item{\textit{In the proof of Theorem 4.1 the authors should be more precise
    about the variables, e.g. $w=(u, v)$. In the second estimate of the proof a
    square is missing.}\\
    Variables $w, \omega$ are now defined at the beginning of the proof. The
    missing power was added. The modifide part of the proof now reads:
    %
    With $w=(u, v)$, $\omega=(\phi, \psi)$ application of the Cauchy-Schwarz 
  inequality yields
  \[
    \begin{split}
      \brack{A w, \omega }_{\Omega} &= \inner{\nabla u }{\nabla \phi}_{\Omega} + 
      \inner{\nabla v}{\nabla \psi}_{\Gamma}
      \\
      & \leq \HnormO{u}\HnormO{\phi} + \HnormG{v}\HnormG{\psi}
      \\
      & \leq \HnormO{u}\HnormO{\phi} + \epsilon^2\HnormG{T_{\Gamma}
      u}\HnormG{\phi} + \HnormG{v}\HnormG{\psi}
      \\
      & \leq \norm{w}_W\norm{\omega}_W.
  \end{split}
  \]
Therefore $A$ is bounded with $\norm{A}=1$ and (A. 1a) holds. The coercivity of 
$A$ on $\ker B$ for (A. 1b) is obtained from
  \[
  \begin{split}
  \inf_{w\in \ker B}\frac{\brack{Aw, w}_{\Omega}}{\norm{w}^2_W} &=
  % 
  \inf_{w\in \ker B}\frac{\HnormO{u}^2 + \HnormG{v}^2}
  {
  \HnormO{u}^2+\epsilon^2\HnormG{T_\Gamma u}^2+\HnormG{v}^2
  }
  \\
  &=
  \inf_{w\in \ker B}\frac{\HnormO{u}^2+\HnormG{v}^2}
  %
  {\HnormO{u}^2 + 2\HnormG{v}^2} \geq \frac{1}{2},
  \end{split}
  \]
  %%%%  
}
%%%%%
  \item{\textit{Please unify citations, e.g. [32, Chapter 5] vs. [24, ch 5].}\\
    Citations have been unified. They now read e.g. [29, ch 9.11].
}
\end{enumerate}

\end{document}
