\documentclass[10pt, a4paper]{article}
\usepackage{amsmath, amsfonts, amssymb}
\usepackage{enumerate}
\usepackage{caption, subcaption, floatrow, subfig}
\usepackage{graphicx}
\usepackage{multirow}
\usepackage{xfrac}
\usepackage{amsmath, amsfonts, amssymb}
\usepackage{tikz}
\usepackage{pgfplots}
\usepackage{pgfplotstable}
\usetikzlibrary{plotmarks, calc}
\usepackage{standalone}
\usepackage{url}

%%%%%%% COLORS
\usepackage{xcolor, colortbl}
\newcolumntype{S}{>{\columncolor{lime!50}}c}         % serial small
\newcolumntype{L}{>{\columncolor{cyan!50}}c}         % serial large
\newcolumntype{P}{>{\columncolor{red!50}}c}          % parallel& 
%%%%%%%
\newcommand{\edit}[3]{\textcolor{blue}{(#1, #2): #3}}

\newcommand{\reals}{\mathbb{R}}
\newcommand{\set}[1]{\{#1\}}
\newcommand{\abs}[1]{\lvert#1\rvert}
\newcommand{\semi}[1]{\lvert#1\rvert}
\newcommand{\norm}[1]{\lVert#1\rVert}
\newcommand{\inner}[2]{\ensuremath{\left(#1, #2\right)}}
\renewcommand{\brack}[1]{\langle#1\rangle}
\newcommand{\average}[1]{\ensuremath{\langle#1\rangle} }
\newcommand{\jump}[1]{\ensuremath{[\![#1]\!]} }
\newcommand{\mat}[1]{\ensuremath{\mathsf{#1}}}
\newcommand{\dual}[1]{\ensuremath{{#1}^{\prime}}}
\renewcommand{\vec}[1]{\mat{#1}}
\newcommand{\supp}{\operatorname{supp}} 

\oddsidemargin=3pt
\textwidth=500pt
\topmargin=3pt

\title{\large{Response to the reviewer}}
\date{}

\begin{document}
\maketitle

We thank the reviewer for his/hers helpful comments on our manuscript. The
comments together with the changes made to the paper which were required to
address them are given below:
\begin{enumerate}
%%%%
  \item{\textit{
The need to solve an eigenvalue problem for the stiffness matrix in the implementation 
of the Q-cap preconditioner is a major weakness of this approach. ... even if an eigenvalue 
problem of this sort is solved only once, it can dominate the computational time altogether.
}
}
%%%%
  \item{\textit{
Not enough attention is given to computational aspects of the approaches taken. It is 
necessary to provide a detailed computational cost analysis of the preconditioners, 
and especially the Q-cap preconditioner. 
}
}
%%%%
  \item{\textit{
While some observations about the spectrum (such as Figure 3.3) are helpful, I wonder if 
it would not be possible to perform a more complete eigenvalue analysis of the preconditioned 
matrix, i.e., to solve the generalized eigenvalue problem analytically to the extent possible 
and attempt to reveal algebraic multiplicity of some of the eigenvalues. (I recognize that this 
may be difficult.)
}
}
%%%%
  \item{\textit{
The paper is fairly sparse with respect to comparison to other effective preconditioning approaches.
  }
}
%%%%
  \item{\textit{
Please provide more information in the numerical experiments. What are the dimensions 
    of the linear systems? What about timings?
    }
}
%%%%
  \item{\textit{
The numerical experiments are, altogether, for rather small problems. Again, given 
that efficiency seems to be an argument made in the paper, it would be necessary 
to demonstrate this efficiency on larger problems. 
  }
}
%%%%
  \item{\textit{
It would also be very good if real world three-dimensional experiments could be presented.
    }
}

\end{enumerate}

\end{document}
